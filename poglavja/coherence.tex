\chapter{Coherence}

To determine the interoperability - ability to interfere - of two EM-waves we compare them at
different \textbf{time} and \textbf{location}. Comparison is done using the \textit{correlation equation} - \ref{eq:correlation}.
\begin{equation}
    \int E_1(x,t) \cdot E_{2}^* (x',t+\tau) d\tau
    \label{eq:correlation}
\end{equation}

We will use some simplifications:
\begin{enumerate}
    \item Absorption works perfectly and is instantaneous
    \item Emitted EM-wave has the same direction, but a random phase
    \item EM-wave is infinite and has a random phase
    \item Wavelength $\lambda$ is not always the same - wavelength fluctuations
    \item Amplitude is not constant - power fluctuations 
\end{enumerate}
We can use the equation for a simplified wave $A(x,t) = A_0 cos(kx - \omega t + \varphi)$,
due to the simplifications 4. and 5., the $\omega t$ becomes the random component.
Connection between this and line-broadening/uncertainty means that the size of phase space in time and frequency
cannot be smaller than $\delta \nu \cdot \tau  \ge \frac{1}{4 \pi}$.
This phase space will not change unless it loses or gains energy.
Product $\delta \nu \cdot  \tau$ is also called \textbf{time-bandwidth product}.

Consequences of this are:
\begin{itemize}
    \item Transitions that are long-lived states provide better wavelength and intensity stability
    \item If $\delta \nu \cdot \tau >> \frac{1}{4\pi}$ wave will have \textbf{strong and random intensity fluctuations} $\rightarrow$ the wave will not interfere with itself
\end{itemize}
The ability of the wave to interfere is called \textbf{coherence}.
Coherence time is the duration during which all the emitted waves can interfere with each other, calculated by \ref{eq:tcoh}.
\begin{equation}
    \tau_{coh} = \frac{1}{\Delta \nu}
    \label{eq:tcoh}
\end{equation}
Where $\Delta \nu$ is measured at $FWHM$. \textbf{Coherence length} is a distance across two waves can interfere, calculated as \ref{eq:lcoh}.
\begin{equation}
    L_{coh} = c \cdot \tau_{coh}
    \label{eq:lcoh}
\end{equation}

Examples of coherence length for different light sources are show in table \ref{tab:clen}.
\begin{table}[h!]
    \centering
    \begin{tabular}{|c|c|c|}
        \hline
        Light source & Wavelength $\lambda$ & Coherence length $L_{coh}$ \\
        \hline
        HeNe single mode & 633 nm & 100 m \\
        \hline
        Argon ion & 488/515 nm & 20 mm \\
        \hline
        GaAIAs, Single mode& 670-905 nm & 3 m\\
        \hline
        Sodium lamp & 2 lines @ 589 nm & 0.6mm \\
        \hline  
        Sunlight & 500 nm & 1 $\mu m$ \\
        \hline     
    \end{tabular}
    \caption{Coherence length}
    \label{tab:clen}
\end{table}

\section{Beam Parameter Product}
Beam parameter product  or BPP is defined as a product of \textbf{beam radius} at waist and \textbf{beam divergence}.

In equation $A(r,t) = A_0 cos(kr - \omega t + \varphi)$ - (\ref{eq:wave}), we have the factors $k,r$. We wish to find a connection between them.
The factors $k$ and $r$ are connected via:
\begin{itemize}
    \item Position-momentum uncertainty: $\Delta x \Delta p_x \ge \frac{h}{4 \pi}$
    \item Fourier transition: space $\leftrightarrow$ spatial frequency (wave vector $k$)
\end{itemize}

Beam parameter product is : $\delta k \cdot \delta r \ge \, const$

Beam parameter product and time-bandwidth product are a consequence of same fundamental laws, meaning that BPP cannot be smaller
than a certain value, which means that light cannot be focused below a certain focal spot size.

%maybe add a picture?

